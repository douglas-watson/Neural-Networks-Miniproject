%%%%%%%%%%%%%%%%%%%%%%%%%%%%%%%%%%%%%%%%%%%%%%%%%%%%%%%%%%%%%%%%%%%%%%%%%%%%%%%%
%2345678901234567890123456789012345678901234567890123456789012345678901234567890
%        1         2         3         4         5         6         7         8

% \documentclass[letterpaper, 10 pt, conference]{ieeeconf}  % Comment this line out
                                                          % if you need a4paper
\documentclass[a4paper, 10pt, conference]{ieeeconf}      % Use this line for a4
                                                          % paper

\IEEEoverridecommandlockouts                              % This command is only
                                                          % needed if you want to
                                                          % use the \thanks command
\overrideIEEEmargins
% See the \addtolength command later in the file to balance the column lengths
% on the last page of the document



\usepackage[pdftex]{graphicx} % modern graphics package
\usepackage[utf8]{inputenc}
\usepackage{amsmath} % assumes amsmath package installed
\usepackage{amssymb}  % assumes amsmath package installed
\usepackage{units}      % pretty units : \unit[val]{dim}
\usepackage{url}        % pretty urls
\usepackage{multirow}   % multiple rows in tables

% Header and footer definitions
\usepackage{fancyhdr}
\renewcommand{\headrulewidth}{0.0pt}
\renewcommand{\footrulewidth}{0.5pt}

\pagestyle{fancyplain}
\fancyhf{}
\rfoot{\fancyplain{}{\thepage}}
\lfoot{\fancyplain{}{\today}}

\pagestyle{fancy}
\fancyhf{}
\rfoot[\today]{\thepage}
\lfoot[\thepage]{\today}
\cfoot[]{}


\graphicspath{{.}{./singlets/}{./search/}{./analysis/}{./CPG/}{./OpenQuestion/}}

\DeclareMathOperator{\sign}{sign}

\title{\LARGE \bf
Locomotion in a feline-inspired quadruped robot
}

\author{Fabian Santi and Douglas Watson, group 3% <-this % stops a space
\thanks{This work is presented for the course ``Biological Models of 
Sensory-Motor Systems'' given by A. J. Ijspeert in the autumn semester of 2010, 
at Ecole Polytechnique Fédérale de Lausanne.  F. Santi is enrolled in the Life 
Sciences program, and D. Watson in the Materials program.}%
}


\begin{document}

\newcommand{\order}[1]{$\cdot 10^{#1}$}
\newcommand{\pvalue}[2]{p-value $< #1 \cdot 10^{#2}$}

\maketitle
\thispagestyle{fancyplain}


%%%%%%%%%%%%%%%%%%%%%%%%%%%%%%%%%%%%%%%%%%%%%%%%%%%%%%%%%%%%%%%%%%%%%%%%%%%%%%%%
\begin{abstract}
The goal of this project was to determine what makes a good trot gait through 
computer simulations of a quadruped robot. The problem was first approached 
using a sine-based controller, by searching for optimal controller parameters 
and studying their effect on the performance of the gait. Performance was 
measured by average speed, stability (average amplitude of oscillations), and 
balance. For this purpose, systematic searches were performed around an initial 
(given) gait. The sine-based controller was then replaced by a central pattern 
generator (CPG), which replicates to some extent the central motor control of 
quadruped animals. The CPG was first studied mathematically, then implemented 
on the simulator. It was then used to study feedback and gait transitions.

\end{abstract}


%%%%%%%%%%%%%%%%%%%%%%%%%%%%%%%%%%%%%%%%%%%%%%%%%%%%%%%%%%%%%%%%%%%%%%%%%%%%%%%%

\setcounter{tocdepth}{2}
\tableofcontents

\input{singlets/intro.tex}

\input{search/search.tex}
\input{analysis/analysis.tex}
\clearpage
\input{CPG/cpg.tex}
\clearpage
\input{OpenQuestion/openquestion.tex}
\input{singlets/conclusion.tex}

\section*{Supplementary material}
\label{sec:supplementary-material}

Supporting videos are available in the bundled archive. Source code and raw 
data, as well as \textsl{R} scripts to interpret the data, are available on 
github: \url{git://github.com/douglas-watson/cheetah-project.git}

\nocite{*}
\bibliographystyle{ieeetr}
\bibliography{cheetah}

\end{document}
